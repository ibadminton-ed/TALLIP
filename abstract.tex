\begin{abstract}
Event extraction is an important technology that underpins many text mining applications. It is a task of extracting events of
certain types and their corresponding arguments of different roles from unstructured texts such as news articles and tweets. 
%Previously, the performance of event extraction  The quality of the features has a significant impact on the resulted . 
State-of-the-art event extraction approaches rely on high quality feature extraction, which requires careful template design
and results of specific semantic or lexical analyzing tools as well as extra thesauruses. 
%   follow a time-consuming manual process for feature selection and tuning. 
%   They heavily rely on the information provided by specific lexical or syntactic analyzing tools. 
As a result, current approaches are hard to generalize,  
%The key challenge here is that the quality of the features strongly depends on the natural language, the structure of the data and the application domain. 
and difficult to adapt to new languages, especially those with limited resources or tools. 
%This means that developers have to spend substantial time in finding a new set of features for each individual application domain or  analyzing tool. 

%Given to the drastic effort  involved in feature tuning, an automatic approach for feature engineering for event extraction will be highly attractive. 
%Archiving automatic feature engineering for Chinese event extraction is an outstanding challenge. 
Different from English, the unique language structures and syntax of Chinese make  prior work on English
 event extraction inapplicable to the new domain.  In this work, we demonstrate, for the first time, it is possible to construct a generalized yet effective
Chinese event extraction model without relying on manually designed feature templates or specific lexical or syntactic
tools. Our novel, broadly applicable approach employs a convolution bidirectional long short-term memory neural
network (\CBiLSTM) to automatically learn a set of effective feature representations. % at both lexical and sentence levels. 
We achieve this by exploiting different neural network components, a convolutional component (\CNN) for prominent lexical features and 
 a bidirectional long short-term memory component (\BiLSTM) capturing sentence level representations. 
To overcome the language-specific challenge of Chinese word segmentation, we combine a conditional random field
(\CRF) layer over a character-level neural network component to allow us tag each character beyond word boundary. %\FIXME{allow us xx??}. 
We evaluate our approach on the ACE 2005 dataset, and experimental results show that our approach can deliver state-of-the-art performance
in both trigger labeling and argument labeling without laborious feature engineering.%, transforming what is potentially months of work into hours.



%In this paper, we
%propose a convolution bidirectional long short-term memory neural network (C-BiLSTM)  model that benefits different
%neural network components to learn both lexical and sentence level feature representations without relying on manually
%designed feature templates or complicated lexical or syntactic tools.  We further propose to combine a CRF layer over a
%character-level neural network components to address the language specific issues in Chinese event extraction.
%Experiments on ACE 2005 dataset show that our approaches can achieve state-of-art performances in both trigger labeling and
%argument role labeling, without feature engineering or extra resources.
\end{abstract}
